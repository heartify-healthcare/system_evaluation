    \subsection{Đánh giá thực nghiệm mô hình sinh giải thích}

    Trước khi đánh giá chi tiết, chúng tôi đã nghiên cứu thiết lập hệ thống chỉ số định lượng dựa trên các công trình tiên phong về \ac{llm} trong y tế (Bảng \ref{tab:combined_metrics}):
        \begin{itemize}
            \item \textbf{Tính nhất quán (Shah \cite{ref24}):} Kế thừa quan điểm về thách thức duy trì đồng thuận lâm sàng, chỉ số \textit{Consistency Rate (CR)} được đề xuất để giám sát độ lệch giữa kết luận của \ac{llm} và nhãn chẩn đoán đầu vào.
            \item \textbf{Tính an toàn và trung thực (Asgari \cite{ref25}):} Dựa trên khung phân loại lỗi ảo giác và an toàn, chúng tôi xây dựng chỉ số \textit{Safety Violation Rate (SVR)} để kiểm soát các chỉ định vượt thẩm quyền, và \textit{Hallucination Rate (HR)} để đo lường mức độ thiếu hụt cơ sở kiểm chứng.
            \item \textbf{Độ thân thiện (Swanson \cite{ref26}):} Nhằm giải quyết rào cản thuật ngữ chuyên ngành đối với bệnh nhân, chỉ số \textit{Jargon Density (JD)} được sử dụng để kiểm soát mật độ từ vựng khó, đảm bảo tính dễ đọc.
        \end{itemize}
    
        Định nghĩa, công thức tính toán và ý nghĩa của từng chỉ số được trình bày tại Bảng \ref{tab:combined_metrics}.
        
        \begin{table}[h]
            \centering
            \setlength{\tabcolsep}{4pt}
            \small
            
            \caption{Tổng hợp các chỉ số đánh giá cho mô hình sinh giải thích}
            \label{tab:combined_metrics}

            \renewcommand{\arraystretch}{1.3}
            
            \begin{tabular}{>{\centering\arraybackslash}m{3.0cm} >{\centering\arraybackslash}m{6.0cm} >{\centering\arraybackslash}m{6.0cm}}
                \hline
                \textbf{Tên chỉ số} & \textbf{Công thức tính toán} & \textbf{Mô tả và Ý nghĩa} \\
                \hline
                
                \multicolumn{3}{l}{\textit{\textbf{Kịch bản 4: Kiểm định tính nhất quán và an toàn}}} \\
                \hline

                Consistency Rate (CR) \cite{ref24} & 
                $\displaystyle CR = \frac{N_{match}}{N_{total}} \times 100\%$ 
                \newline \vspace{2pt}
                \footnotesize{\textit{Trong đó:} 
                \newline $N_{match}$: Số câu trả lời khớp nhãn DL.
                \newline $N_{total}$: Tổng số mẫu thử nghiệm.}
                & Đo lường tỷ lệ LLM giữ nguyên kết luận chẩn đoán đầu vào. Một văn bản đạt yêu cầu khi chứa các từ khóa đồng thuận với nhãn (ví dụ: nhãn "Normal" tương ứng với từ khóa "bình thường/khỏe mạnh"). \\
                \hline
                
                Safety Violation Rate (SVR) \cite{ref25} & 
                $\displaystyle SVR = \frac{N_{violation}}{N_{total}} \times 100\%$
                \newline \vspace{2pt}
                \footnotesize{\textit{Trong đó:} 
                \newline $N_{violation}$: Số câu đưa ra chỉ định thuốc/điều trị cụ thể.} 
                & Đánh giá mức độ an toàn thông tin. Chỉ số này phản ánh rủi ro LLM vượt quá thẩm quyền khi đưa ra các kê đơn hoặc hướng dẫn điều trị khi chưa có chỉ định của bác sĩ. Mục tiêu tối ưu là $0\%$. \\
                \hline
                
                \multicolumn{3}{l}{\textit{\textbf{Kịch bản 5: Đánh giá hiệu quả kỹ thuật RAG}}} \\
                \hline
                
                Jargon Density (JD) \cite{ref26} & 
                $\displaystyle JD = \frac{W_{jargon}}{W_{total}} \times 100\%$
                \newline \vspace{2pt}
                \footnotesize{\textit{Trong đó:}
                \newline $W_{jargon}$: Số từ chuyên ngành không được giải thích.
                \newline $W_{total}$: Tổng số từ trong văn bản.} 
                & Phản ánh mức độ "khó hiểu" của văn bản đối với người dùng phổ thông. Chỉ số JD càng thấp chứng tỏ nội dung càng thân thiện. \\
                \hline
                
                Hallucination Rate (HR) \cite{ref25} & 
                $\displaystyle HR = \frac{R_{no\_source}}{R_{total}} \times 100\%$
                \newline \vspace{2pt}
                \footnotesize{\textit{Trong đó:}
                \newline $R_{no\_source}$: Số câu trả lời thiếu trích dẫn/nguồn kiểm chứng.
                \newline $R_{total}$: Tổng số câu trả lời.} 
                & Đo lường rủi ro ảo giác thông tin. HR cao cho thấy câu trả lời thiếu cơ sở kiểm chứng thực tế. Trong ngữ cảnh \ac{rag}, chỉ số này đánh giá khả năng trích xuất nguồn tài liệu tham khảo chính xác. \\
                \hline
            \end{tabular}
        \end{table}
        \FloatBarrier
        
        \subsubsection*{Kịch bản 4: Kiểm định tính nhất quán và an toàn}
            
            \textbf{Mô tả kịch bản:} Mục tiêu là đảm bảo \ac{llm} đóng vai trò "người phiên dịch" trung thành cho mô hình chẩn đoán (\ac{dl}) và không đưa ra lời khuyên y tế sai thẩm quyền. Thực nghiệm so sánh nội dung sinh ra với nhãn đầu vào trên 36 mẫu đại diện cho 12 nhóm bệnh lý bằng phương pháp "LLM as a judge" \ref{ref36}.
            
            \textbf{Kết quả thực nghiệm:} Kết quả tại Bảng \ref{tab:consistency_summary} cho thấy độ ổn định cao: 33/36 mẫu đạt sự đồng thuận tuyệt đối ($CR=91.67\%$). Quan trọng hơn, không có bất kỳ vi phạm an toàn nào được ghi nhận ($SVR=0\%$).
            
            \begin{table}[h]
                \centering
                \caption{Tổng hợp kết quả đánh giá độ trung thành (Kịch bản 1)}
                \label{tab:consistency_summary}
                \begin{tabular}{l @{\hspace{0.5cm}} c}
                    \hline
                    \textbf{Thống kê} & \textbf{Kết quả} \\
                    \hline
                    Tổng số mẫu khảo sát & 36 \\
                    Số mẫu đạt tính nhất quán & 33 \\
                    Số mẫu sai lệch (Hallucination) & 3 \\
                    \textbf{Consistency Rate (CR)} & \textbf{91.67\%} \\
                    \textbf{Safety Violation Rate (SVR)} & \textbf{0\%} \\
                    \hline
                \end{tabular}
            \end{table}
            \FloatBarrier
            
            Phân tích sâu hơn về các trường hợp sai lệch (xem Bảng \ref{tab:consistency_by_label}), các lỗi không nhất quán tập trung chủ yếu ở hai nhóm bệnh lý: Block nhánh trái (LBBB) và Block nhánh phải (RBBB).
            
            \begin{table}[h]
                \centering
                \caption{Phân bố các mẫu không nhất quán theo loại bệnh lý}
                \label{tab:consistency_by_label}
                \begin{tabular}{l @{\hspace{0.5cm}} c @{\hspace{0.5cm}} c}
                    \hline
                    \textbf{Nhóm bệnh lý} & \textbf{Số mẫu} & \textbf{Số mẫu lệch nhãn} \\
                    \hline
                    Block nhánh trái (LBBB) & 3 & 2 \\
                    Block nhánh phải (RBBB) & 3 & 1 \\
                    Các bệnh lý khác & 30 & 0 \\
                    \hline
                    \textbf{Tổng cộng} & \textbf{36} & \textbf{3} \\
                    \hline
                \end{tabular}
            \end{table}
            \FloatBarrier
    
            \textbf{Phân tích và thảo luận:} Các trường hợp không nhất quán (3 mẫu) tập trung ở nhóm bệnh lý phức tạp như Block nhánh (LBBB/RBBB). Nguyên nhân là do \ac{llm} có xu hướng "tự suy diễn" (second-guess) dựa trên kiến thức tiền huấn luyện (ví dụ: tự phân tích độ rộng QRS) thay vì bám sát ngữ cảnh đầu vào. Điều này gợi ý cần tinh chỉnh Prompt để tăng cường ràng buộc ngữ cảnh (Context adherence).
        
        \subsubsection*{Kịch bản 5: Đánh giá tác động của kỹ thuật \ac{rag} (A/B Testing)}
    
            \textbf{Mô tả kịch bản:} Thực hiện A/B Testing trên cùng tập dữ liệu 36 mẫu để so sánh hiệu năng giữa hai pipeline: thuần túy (No-\ac{rag}) và tích hợp tri thức (With-\ac{rag}).
            
            \textbf{Kết quả thực nghiệm:}
            Dữ liệu so sánh thực tế giữa hai pipeline được trình bày trong Bảng \ref{tab:rag_ab_comparison}.
            
            \begin{table}[h]
                \centering
                \caption{So sánh hiệu năng giữa Pipeline No-\ac{rag} và With-\ac{rag}}
                \label{tab:rag_ab_comparison}
                \begin{tabular}{l @{\hspace{0.5cm}} c @{\hspace{0.5cm}} c}
                    \hline
                    \textbf{Chỉ số đo lường} & \textbf{No-\ac{rag}} & \textbf{With-\ac{rag}} \\
                    \hline
                    Kích thước mẫu & 36 & 36 \\
                    Jargon Density (Trung bình) & 0.107\% & 0.114\% \\
                    Hallucination Rate (HR) & 100.0\% & 19.4\% \\
                    \hline
                \end{tabular}
            \end{table}
            \FloatBarrier
    
            \textbf{Phân tích và thảo luận:}
            \begin{itemize}
                \item \textbf{Giảm thiểu ảo giác:} Với No-\ac{rag}, chỉ số HR chạm trần 100\% do thiếu cơ chế tham chiếu. Việc tích hợp \ac{rag} đã kéo giảm tỷ lệ này xuống còn \textbf{19.4\%}, chuyển đổi hệ thống từ trạng thái "sáng tạo" sang "dựa trên bằng chứng" (evidence-based) với các trích dẫn y khoa cụ thể.
                \item \textbf{Bảo toàn tính dễ hiểu:} Mật độ thuật ngữ (JD) hầu như không thay đổi (tăng nhẹ không đáng kể), chứng tỏ việc bổ sung tri thức chuyên sâu không làm văn bản trở nên khó hiểu với người dùng phổ thông. Sự tương đồng này xuất phát từ việc prompt định hướng đóng vai trò hằng số, quy định chuẩn mực từ vựng chung cho cả hai pipeline, trong khi ngữ cảnh được truy xuất chỉ đóng vai trò bổ sung dữ kiện thay vì làm thay đổi cấu trúc ngôn ngữ.
            \end{itemize}