\subsection{Đánh giá thực nghiệm mô hình sinh giải thích}

    Trước khi đánh giá chi tiết, chúng tôi đã nghiên cứu thiết lập hệ thống chỉ số định lượng dựa trên các công trình tiên phong về \ac{llm} trong y tế (Bảng \ref{tab:combined_metrics}):
        \begin{itemize}
            \item \textbf{Tính nhất quán (Shah \cite{ref24}):} Kế thừa quan điểm về thách thức duy trì đồng thuận lâm sàng, chỉ số \textit{Consistency Rate (CR)} được đề xuất để giám sát độ lệch giữa kết luận của \ac{llm} và nhãn chẩn đoán đầu vào[cite: 1, 2].
            \item \textbf{Tính an toàn và trung thực (Asgari \cite{ref25}):} Dựa trên khung phân loại lỗi ảo giác và an toàn, chúng tôi xây dựng chỉ số \textit{Safety Violation Rate (SVR)} để kiểm soát các chỉ định vượt thẩm quyền, và \textit{Hallucination Rate (HR)} để đo lường mức độ thiếu hụt cơ sở kiểm chứng[cite: 2].
            \item \textbf{Độ thân thiện (Swanson \cite{ref26}):} Nhằm giải quyết rào cản thuật ngữ chuyên ngành đối với bệnh nhân, chỉ số \textit{Jargon Density (JD)} được sử dụng để kiểm soát mật độ từ vựng khó, đảm bảo tính dễ đọc[cite: 3].
        \end{itemize}
    
        Định nghĩa, công thức tính toán và ý nghĩa của từng chỉ số được trình bày tại Bảng \ref{tab:combined_metrics}.

        % [GIỮ NGUYÊN BẢNG tab:combined_metrics NHƯ CŨ VÌ KHÔNG ĐỔI ĐỊNH NGHĨA]
        \begin{table}[h]
            \centering
            \setlength{\tabcolsep}{4pt}
            \small
            \caption{Tổng hợp các chỉ số đánh giá cho mô hình sinh giải thích}
            \label{tab:combined_metrics}
            \renewcommand{\arraystretch}{1.3}
            \begin{tabular}{>{\centering\arraybackslash}m{3.0cm} >{\centering\arraybackslash}m{6.0cm} >{\centering\arraybackslash}m{6.0cm}}
                \hline
                \textbf{Tên chỉ số} & \textbf{Công thức tính toán} & \textbf{Mô tả và Ý nghĩa} \\
                \hline
                \multicolumn{3}{l}{\textit{\textbf{Kịch bản 4: Kiểm định tính nhất quán và an toàn}}} \\
                \hline
                Consistency Rate (CR) \cite{ref24} & 
                $\displaystyle CR = \frac{N_{match}}{N_{total}} \times 100\%$ 
                \newline \vspace{2pt}
                \footnotesize{\textit{Trong đó:} 
                \newline $N_{match}$: Số câu trả lời khớp nhãn DL.
                \newline $N_{total}$: Tổng số mẫu thử nghiệm.}
                & Đo lường tỷ lệ LLM giữ nguyên kết luận chẩn đoán đầu vào. Một văn bản đạt yêu cầu khi chứa các từ khóa đồng thuận với nhãn. \\
                \hline
                Safety Violation Rate (SVR) \cite{ref25} & 
                $\displaystyle SVR = \frac{N_{violation}}{N_{total}} \times 100\%$
                \newline \vspace{2pt}
                \footnotesize{\textit{Trong đó:} 
                \newline $N_{violation}$: Số câu đưa ra chỉ định thuốc/điều trị cụ thể.} 
                & Đánh giá mức độ an toàn thông tin. Mục tiêu tối ưu là $0\%$. \\
                \hline
                \multicolumn{3}{l}{\textit{\textbf{Kịch bản 5: Đánh giá hiệu quả kỹ thuật RAG}}} \\
                \hline
                Jargon Density (JD) \cite{ref26} & 
                $\displaystyle JD = \frac{W_{jargon}}{W_{total}} \times 100\%$
                \newline \vspace{2pt}
                \footnotesize{\textit{Trong đó:}
                \newline $W_{jargon}$: Số từ chuyên ngành không được giải thích.
                \newline $W_{total}$: Tổng số từ trong văn bản.} 
                & Phản ánh mức độ "khó hiểu" của văn bản. Chỉ số JD càng thấp chứng tỏ nội dung càng thân thiện. \\
                \hline
                Hallucination Rate (HR) \cite{ref25} & 
                $\displaystyle HR = \frac{R_{no\_source}}{R_{total}} \times 100\%$
                \newline \vspace{2pt}
                \footnotesize{\textit{Trong đó:}
                \newline $R_{no\_source}$: Số câu trả lời thiếu trích dẫn/nguồn kiểm chứng.
                \newline $R_{total}$: Tổng số câu trả lời.} 
                & Đo lường rủi ro ảo giác thông tin. HR cao cho thấy câu trả lời thiếu cơ sở kiểm chứng thực tế. \\
                \hline
            \end{tabular}
        \end{table}
        \FloatBarrier
        
        \subsubsection*{Kịch bản 4: Kiểm định tính nhất quán và an toàn (LLM as a Judge)}
            
            \textbf{Mô tả kịch bản:} Mục tiêu là đảm bảo \ac{llm} đóng vai trò "người phiên dịch" trung thành cho mô hình chẩn đoán (\ac{dl}) và tuyệt đối tuân thủ các quy tắc an toàn y tế. Thay vì đánh giá thủ công, chúng tôi áp dụng phương pháp "LLM as a judge" \cite{ref36} để so sánh hiệu năng của 5 mô hình ngôn ngữ lớn tiên tiến nhất hiện nay (GPT-5 series, Claude Sonnet 4.5, và Gemini Pro) trên tập dữ liệu kiểm thử.
            
            \textbf{Kết quả thực nghiệm:} Kết quả tổng hợp tại Bảng \ref{tab:consistency_comparison} cho thấy sự vượt trội của mô hình Claude Sonnet 4.5 và tính an toàn tuyệt đối của toàn bộ hệ thống.

            \begin{table}[h]
                \centering
                \caption{So sánh tính Nhất quán (CR) và An toàn (SVR) giữa các mô hình}
                \label{tab:consistency_comparison}
                \renewcommand{\arraystretch}{1.2}
                \begin{tabular}{l c c}
                    \hline
                    \textbf{Mô hình} & \textbf{Consistency Rate (CR)} & \textbf{Safety Violation Rate (SVR)} \\
                    \hline
                    GPT-5.0 & 91.67\% & 0\% \\
                    GPT-5.1 & 91.67\% & 0\% \\
                    GPT-5.2 & 91.67\% & 0\% \\
                    \textbf{Claude Sonnet 4.5} & \textbf{94.44\%} & \textbf{0\%} \\
                    Gemini Pro & 91.67\% & 0\% \\
                    \hline
                    \textit{Trung bình} & \textit{92.22\%} & \textit{0\%} \\
                    \hline
                \end{tabular}
            \end{table}
            \FloatBarrier
            
            \textbf{Phân tích và thảo luận:}
            \begin{itemize}
                \item \textbf{Độ trung thành cao:} Hầu hết các mô hình đều đạt tỷ lệ nhất quán $CR \approx 91.67\%$ (33/36 mẫu khớp nhãn). Đặc biệt, \textbf{Claude Sonnet 4.5} cho thấy khả năng bám sát ngữ cảnh tốt nhất với $CR = 94.44\%$. Các trường hợp sai lệch còn lại chủ yếu tập trung ở nhóm bệnh lý phức tạp (như Block nhánh) nơi mô hình có xu hướng tự phân tích lại tín hiệu thay vì chấp nhận nhãn đầu vào.
                \item \textbf{An toàn tuyệt đối:} Một tín hiệu rất tích cực là chỉ số $SVR$ đạt $0\%$ trên tất cả các mô hình thử nghiệm. Điều này chứng minh cơ chế Prompt Engineering hiện tại đã ngăn chặn hiệu quả việc AI đưa ra các chỉ định thuốc hoặc phác đồ điều trị trái thẩm quyền.
            \end{itemize}

        \subsubsection*{Kịch bản 5: Đánh giá tác động của kỹ thuật \ac{rag} (No-RAG vs With-RAG)}
    
            \textbf{Mô tả kịch bản:} Chúng tôi thực hiện so sánh đối chứng (A/B Testing) trên 5 mô hình để đo lường tác động của việc tích hợp tri thức (\ac{rag}) đối với hai khía cạnh: rủi ro ảo giác (Hallucination) và độ thân thiện của văn bản (Jargon Density).
            
            \textbf{Kết quả thực nghiệm:} Dữ liệu so sánh chi tiết được trình bày trong Bảng \ref{tab:rag_comparison_all}.

            \begin{table}[h]
                \centering
                \small
                \caption{Tác động của RAG lên chỉ số Ảo giác (HR) và Thuật ngữ (JD)}
                \label{tab:rag_comparison_all}
                \setlength{\tabcolsep}{3pt}
                \begin{tabular}{l | c c | c c}
                    \hline
                    \multirow{2}{*}{\textbf{Mô hình}} & \multicolumn{2}{c|}{\textbf{Hallucination Rate (HR)}} & \multicolumn{2}{c}{\textbf{Jargon Density (JD)}} \\
                    \cline{2-5}
                     & No-RAG & \textbf{With-RAG} & No-RAG & With-RAG \\
                    \hline
                    GPT-5.0 & 100\% & 19.44\% & 0.27\% & 0.39\% \\
                    GPT-5.1 & 100\% & 22.22\% & 0.03\% & 0.05\% \\
                    GPT-5.2 & 100\% & 19.44\% & 0.11\% & 0.11\% \\
                    Claude Sonnet 4.5 & 100\% & 22.22\% & 0.93\% & 0.88\% \\
                    Gemini Pro & 100\% & 19.44\% & 0.09\% & 0.06\% \\
                    \hline
                \end{tabular}
            \end{table}
            \FloatBarrier
    
            \textbf{Phân tích và thảo luận:}
            \begin{itemize}
                \item \textbf{Giảm thiểu ảo giác triệt để:} Ở chế độ No-RAG, chỉ số HR chạm trần 100\% ở tất cả các mô hình do thiếu cơ chế tham chiếu nguồn. Việc tích hợp \ac{rag} đã kéo giảm đáng kể tỷ lệ này xuống mức trung bình khoảng \textbf{20.55\%}. Các mô hình như GPT-5.0, GPT-5.2 và Gemini Pro đạt hiệu suất tốt nhất với HR thấp nhất ($19.44\%$). Điều này xác nhận hệ thống đã chuyển dịch thành công từ trạng thái "sáng tạo tự do" sang "dựa trên bằng chứng" (evidence-based).
                
                \item \textbf{Kiểm soát độ khó hiểu:} Mật độ thuật ngữ (JD) có sự biến động nhẹ giữa các mô hình nhưng đều duy trì ở mức rất thấp (dưới 1\%). Đáng chú ý, Claude Sonnet 4.5 có xu hướng sử dụng từ vựng chuyên sâu hơn ($JD \approx 0.9\%$) so với các dòng GPT ($JD \approx 0.1-0.3\%$). Tuy nhiên, việc áp dụng RAG hầu như không làm tăng đáng kể độ khó của văn bản, đảm bảo tính thân thiện với người dùng phổ thông.
            \end{itemize}